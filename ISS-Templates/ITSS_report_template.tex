% Template; to be used with:
%          spconf.sty  - ICASSP/ICIP LaTeX style file, and
%          IEEEbib.bst - IEEE bibliography style file.
% --------------------------------------------------------------------------
\documentclass{article}
\usepackage{spconf,amsmath,graphicx}

% Example definitions.
% --------------------
\def\x{{\mathbf x}}
\def\L{{\cal L}}

% Title.
% ------
\title{Intelligent Sensing Systems Practice Module Report Template}
%
% Single address.
% ---------------
\name{First1 Last1,  First2 Last2}
\address{Institute of Systems Science, National University of Singapore, Singapore 119615}

\begin{document}
%\ninept
%
\maketitle
%

\begin{abstract}

The abstract should consist of one paragraph describing the motivation for your project and a high-level explanation of the methodology you used and the results obtained. Note: this project report template is modified from the report template used in Stanford University CS230, https://cs230.stanford.edu/. Please note that this template is in a two-column format. Page limit: 8-10 pages including everything.

\end{abstract}
%

\begin{keywords}
Keyword one, Keyword two, Keyword three, Keyword four,
\end{keywords}
%
\section{Introduction}
\label{sec:intro}

Explain the problem and why it is important. Discuss your motivation for pursuing this problem. Give some background if necessary.

Clearly state what the input and output are. Be very explicit: “The input to our algorithm is an {image, video, RGB-D, audio}. We then use a {neural network, etc.} to output a predicted {age, facial expression, action music genre, etc.}.” This is very important since different teams have different inputs/outputs spanning different application domains. Being explicit about this makes it easier for readers.

Add one paragraph to clearly state the contribution/highlights (i.e., the selling point) of your project. The contribution/highlights of this project can be summarized as follows.
\begin{itemize}
\item First, blah, blah.
\item Second, blah, blah.
\end{itemize}

\section{Literature review}
You should find relevant references (e.g., papers, survey, industrial products), group them into categories based on their approaches, and discuss their strengths and weaknesses, as well as how they are similar to and differ from your work.

\subsection{Related research works}

\subsection{Related commercial solutions}

\section{Dataset}
Describe your dataset: how many categories, how many instances. Include a citation on where you obtained your dataset from. If you collected yourself, described how the data was captured. How many training/validation/test examples do you have?  What is the resolution of your images?  Try to include examples of your data in the report (e.g. include an image, show a waveform, etc.).


\section{Proposed approach}
Describe your proposed system. You might want to use a system architecture or flow chart to illustrate your proposed system. For each module, give a detailed description of how it works. Even if you use the pre-trained model in some modules, provide a description.

\section{Experimental results}

\subsection{Implementation details}

What about normalization or data augmentation? Is there any pre-processing you did? What (hyper)parameters are used? What are model training details, such as optimizer, epoch, batch size, learning rate, GPU specification, etc?

\subsection{Performance metrics}
What your primary metrics are: accuracy, precision, etc? Provide equations for the metrics if necessary.

\subsection{Experimental results}
For results, you want to have a mixture of tables and plots. You need to have both quantitative and qualitative results! Include visualizations of results.

Experiments are required to provide performance evaluation by comparing your system with either the existing systems (e.g., reference code) or different choices of modules inside your system (e.g., backbone, model hyper-parameters).

\subsection{Ablation study}
It would be good to conduct ablation studies to evaluate how various components of your framework contribute to its final performance.

\subsection{Discussions and limitations}
It would be good to include some examples of where your approach failed and a discussion of why certain approaches failed or succeeded. 

Some Latex examples are provided as follows. An inline equation is $a+b=c$. An example of a one-column figure is provided in Figure \ref{figure2}.

\begin{equation}\label{equation block model}
B_{r,c}=\sum\{f(i,j)|(i,j)\in \Omega_{r,c}\}.
\end{equation}
\begin{equation}\label{equation 1}
\sum_{x}=a+b+\hat{c},
\end{equation}

\begin{figure}[tbh]
    \centerline{\begin{tabular}{cc|c}
        \includegraphics[width=3cm]{ISS.png}
        &\includegraphics[width=3cm]{ISS.png}& text\\
    (a) & (b) & (c)
    \end{tabular}}
    \caption{Test figure (single column).\label{figure2}}
\end{figure}

\begin{table}[tbh]
\caption{The performance comparison.}\label{table1} \centerline{
    \begin{tabular}{clc|r}
    \hline\hline
    Approach & Ref. \cite{He:16CVPR} & Ref. \cite{Nguyen:18IET} & Proposed approach\\
    Metric A & $0.8181$ & $0.9171$ & $0.9616$ \\\hline
    Metric B & $0.8236$ & $0.7654$ & $0.8615$ \\\hline
    \end{tabular}
    }
\end{table}


\section{Conclusions and future work}
Summarize your report and reiterate key points. For future work, if you had more time, more team members, or more computational resources, what would you explore?

It would be good to have around 15-20 references for the whole report.

\section{Author contributions}
This section should describe the contributions of each team member to the project.

\begin{itemize}
\item First team member, blah, blah.
\item Second team member, blah, blah.
\end{itemize}


\bibliographystyle{IEEEbib}
\bibliography{references}

\end{document}
